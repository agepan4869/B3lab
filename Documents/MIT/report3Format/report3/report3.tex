\documentclass[submit]{literacy}
%\documentclass{ipsj}
\usepackage{graphicx}
\usepackage{latexsym}
\usepackage{ascmac}

\def\Underline{\setbox0\hbox\bgroup\let\\\endUnderline}
\def\endUnderline{\vphantom{y}\egroup\smash{\underline{\box0}}\\}
\def\|{\verb|}

\setcounter{巻数}{1}
\setcounter{号数}{1}
\setcounter{page}{1}
\setcounter{month}{8}
\setcounter{year}{2017}

\受付{2017}{8}{1}
\採録{2015}{8}{1}

\begin{document}

\title{ わかりやすいタイトルつける }
% \etitle{How to Prepare Your Paper for IPSJ Journal \\ (ipsj.cls version 2.0)}
\affiliate{GP}{グループ999(ここに自分のグループ番号を記入)}

\author{氏名}{ローマ字氏名}{GP}[学籍番号(担当したsection)]
\author{情報 太郎}{Taro Joho}{GP}[123456(2章)]
\author{室蘭 花子}{Hanako Muroran}{GP}[78901(1章)]
\author{工業 次郎}{Jiro Kougyo}{GP}[234567(3章)]

\begin{abstract}
情報リテラシー演習 第3回レポート:グループ999(ここに自分のグループ番号を記入)\\

レポートを書くにあたって工夫した点,苦労した点,
本講義で学んだことなどを簡潔に250文字程度で記入
\end{abstract}

\maketitle

\section{レポートの体裁}
レポートの体裁はこの「ひな型」に基づいて整えてください。
ここで、ひな型のファイル名は \verb+report3-gXX.tex+ となっていますが、実際には
 \verb+XX+ の部分を各グループの番号に置き換えてください。
レポートの内容については「情報リテラシー演習の手引き」に書いてある必須要件を必ず満たすように注意してください。

\section{PDF版の提出方法}
PDFファイルは\LaTeX ソースファイルから以下の要領で作成できます。
\begin{screen}
\small
\begin{verbatim}
jxxxxxxxx@ubuntu:~$ platex report3-gXX.tex
jxxxxxxxx@ubuntu:~$ dvipdfmx report3-gXX.dvi
\end{verbatim}
\normalsize
\end{screen}
この結果として、\verb+report3-gXX.pdf+ ファイルができあがります。
レポートが完成したら、PDFファイルを印刷し、提出に備えてください。

\section{\LaTeX ファイルの分割作成法}
手分けして共同で\LaTeX 文章を作成する場合は、担当部分ごとにファイルを分割して作成することになります。
たとえば、このひな型のように、各自の担当部分がそれぞれ異なるファイル \verb+report3-taro.tex+、\verb+report3-jiro.tex+、\verb+report3-saburo.tex+ として作成されることになります。
このひな形では、これらのファイルを親ファイル \verb+report3-gXX.tex+ の \verb+\input+ 命令を用いて統合しています。
\verb+\input+ 命令を使うと、その場所に \verb+{  }+ で指定したファイルが読み込まれ、コンパイルされます。

ここで、下の例のように、他人のファイルを読み込む \verb+\input+コマンドを \verb+%+ 記号でコメントアウトすれば、自分の担当部分だけをコンパイルして確認することもできます。

\small
\begin{itembox}{次郎の分だけinputする例}
\begin{verbatim}
\documentclass[a4j,twocolumn]{jarticle}
\usepackage{ascmac}
\usepackage{graphicx}

\title{情報リテラシー演習 レポート3 \LaTeX ひな形}
\author{情報リテラシー演習 担当スタッフ}
\date{20XX年X月XX日}

\begin{document}
\maketitle

%\section{レポートの体裁}
レポートの体裁はこの「ひな型」に基づいて整えてください。
ここで、ひな型のファイル名は \verb+report3-gXX.tex+ となっていますが、実際には
 \verb+XX+ の部分を各グループの番号に置き換えてください。
レポートの内容については「情報リテラシー演習の手引き」に書いてある必須要件を必ず満たすように注意してください。

\section{PDF版の提出方法}
PDFファイルは\LaTeX ソースファイルから以下の要領で作成できます。
\begin{screen}
\small
\begin{verbatim}
jxxxxxxxx@ubuntu:~$ platex report3-gXX.tex
jxxxxxxxx@ubuntu:~$ dvipdfmx report3-gXX.dvi
\end{verbatim}
\normalsize
\end{screen}
この結果として、\verb+report3-gXX.pdf+ ファイルができあがります。
レポートが完成したら、PDFファイルを印刷し、提出に備えてください。

%\section{\LaTeX ファイルの分割作成法}
手分けして共同で\LaTeX 文章を作成する場合は、担当部分ごとにファイルを分割して作成することになります。
たとえば、このひな型のように、各自の担当部分がそれぞれ異なるファイル \verb+report3-taro.tex+、\verb+report3-jiro.tex+、\verb+report3-saburo.tex+ として作成されることになります。
このひな形では、これらのファイルを親ファイル \verb+report3-gXX.tex+ の \verb+\input+ 命令を用いて統合しています。
\verb+\input+ 命令を使うと、その場所に \verb+{  }+ で指定したファイルが読み込まれ、コンパイルされます。

ここで、下の例のように、他人のファイルを読み込む \verb+\input+コマンドを \verb+%+ 記号でコメントアウトすれば、自分の担当部分だけをコンパイルして確認することもできます。

\small
\begin{itembox}{次郎の分だけinputする例}
\begin{verbatim}
\documentclass[a4j,twocolumn]{jarticle}
\usepackage{ascmac}
\usepackage{graphicx}

\title{情報リテラシー演習 レポート3 \LaTeX ひな形}
\author{情報リテラシー演習 担当スタッフ}
\date{20XX年X月XX日}

\begin{document}
\maketitle

%\section{レポートの体裁}
レポートの体裁はこの「ひな型」に基づいて整えてください。
ここで、ひな型のファイル名は \verb+report3-gXX.tex+ となっていますが、実際には
 \verb+XX+ の部分を各グループの番号に置き換えてください。
レポートの内容については「情報リテラシー演習の手引き」に書いてある必須要件を必ず満たすように注意してください。

\section{PDF版の提出方法}
PDFファイルは\LaTeX ソースファイルから以下の要領で作成できます。
\begin{screen}
\small
\begin{verbatim}
jxxxxxxxx@ubuntu:~$ platex report3-gXX.tex
jxxxxxxxx@ubuntu:~$ dvipdfmx report3-gXX.dvi
\end{verbatim}
\normalsize
\end{screen}
この結果として、\verb+report3-gXX.pdf+ ファイルができあがります。
レポートが完成したら、PDFファイルを印刷し、提出に備えてください。

%\section{\LaTeX ファイルの分割作成法}
手分けして共同で\LaTeX 文章を作成する場合は、担当部分ごとにファイルを分割して作成することになります。
たとえば、このひな型のように、各自の担当部分がそれぞれ異なるファイル \verb+report3-taro.tex+、\verb+report3-jiro.tex+、\verb+report3-saburo.tex+ として作成されることになります。
このひな形では、これらのファイルを親ファイル \verb+report3-gXX.tex+ の \verb+\input+ 命令を用いて統合しています。
\verb+\input+ 命令を使うと、その場所に \verb+{  }+ で指定したファイルが読み込まれ、コンパイルされます。

ここで、下の例のように、他人のファイルを読み込む \verb+\input+コマンドを \verb+%+ 記号でコメントアウトすれば、自分の担当部分だけをコンパイルして確認することもできます。

\small
\begin{itembox}{次郎の分だけinputする例}
\begin{verbatim}
\documentclass[a4j,twocolumn]{jarticle}
\usepackage{ascmac}
\usepackage{graphicx}

\title{情報リテラシー演習 レポート3 \LaTeX ひな形}
\author{情報リテラシー演習 担当スタッフ}
\date{20XX年X月XX日}

\begin{document}
\maketitle

%\input{report3-taro.tex}
\input{report3-jiro.tex}
%\input{report3-saburo.tex}

\appendix
\section{担当者一覧}

       (途中省略)

\end{document}
\end{verbatim}
\end{itembox}
\normalsize


\appendix
\section{担当者一覧}

       (途中省略)

\end{document}
\end{verbatim}
\end{itembox}
\normalsize


\appendix
\section{担当者一覧}

       (途中省略)

\end{document}
\end{verbatim}
\end{itembox}
\normalsize


\end{document}
